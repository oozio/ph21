\documentclass[]{scrartcl}

%opening
\title{PH21 Assignment 6 report}
\author{Helen Xue}

\begin{document}

\maketitle

\section{Method for finding Principal Components}
\par Following the treatment in the Shiens paper, we can find the principal component for a set of generated 2D data as follows:
\begin{enumerate}
	\item Generate the data: let x be uniform, and y be a linear function of x with some error. The error is generated using a Gaussian random variable with a mean of 1 and a standard deviation of 0.5. Here I generated two arrays of size 10.
	\item Combine x and y into one matrix xy and normalize the matrix to have 0 mean.
	\item Use numpy.cov to find the covariance matrix for xy
	\item Use numpy.linalg.eig to find the eigenvalues and eigenvectors of the covariance matrix.
	\item Use numpy.transpose to get the principal components from the eigenvectors. 	
\end{enumerate}

\par The process is similar for a higher-dimensional data set (here there are 3 dependent variables):
\begin{enumerate}
	\item Generate the data: let x be uniform, and y1,y2,y3 be linear function of x with some error. The error is generated using a Gaussian random variable with a mean of 1 and a standard deviation of 0.5. Here I generated two arrays of size 10.
	\item Combine x and y1,y2,y3 into one matrix xy and normalize the matrix to have 0 mean.
	\item Use numpy.cov to find the covariance matrix for xy
	\item Use numpy.linalg.eig to find the eigenvalues and eigenvectors of the covariance matrix.
	\item Use numpy.transpose to get the principal components from the eigenvectors. 	
\end{enumerate}

\section{Results}
\par The smaller the eigenvalue, the less important the associated principal component: the eigenvalue lets us know the spread, and the lower the eigenvalue, the noisier that particular degree of freedom is.
\par 2D case:

\par eigenvalue: 0.0009190906545804012; pc: [-0.83392113  0.55188364]
\par eigenvalue: 28.920427851861454; pc: [-0.55188364 -0.83392113]

\bigskip
\par 4D case:
\par eigenvalue: 23.40471194722344; pc: [0.61791707 0.66880666 0.07743473 0.4060542 ]
\par eigenvalue: 0.065051557053434; pc: [-0.75523464  0.5628972  -0.20985111  0.26216388]
\par eigenvalue: 0.178672491231527; pc: [ 0.18695206 -0.20384939 -0.93324546  0.2292319 ]
\par eigenvalue: 0.259501559246700; pc: [-0.11334932 -0.44078319  0.28110368  0.8448922 ]
\bigskip

We see that the noise-free channel had a high corresponding eigenvalue. 

\end{document}
